%----------------------------------------------------------------------------------------
%	PACKAGES AND DOCUMENT CONFIGURATIONS
%----------------------------------------------------------------------------------------

\documentclass{article}
\usepackage{graphicx} % Required for the inclusion of images
\usepackage{subfigure} % Required for the inclusion of images
\usepackage{natbib} % Required to change bibliography style to APA
\usepackage{amsmath} % Required for some math elements 



%\usepackage{times} % Uncomment to use the Times New Roman font

%----------------------------------------------------------------------------------------
%	DOCUMENT INFORMATION
%----------------------------------------------------------------------------------------

\title{\textbf{Project 1: Optimizing the Performance of a Pipelined Processor}} % Title

\author{000, 赵梓淇, bugenzhao@sjtu.edu.cn \\
        001, 赵一民, doctormin@sjtu.edu.cn \\ } % Author name and email

\date{\today} % Date for the report

\begin{document}

\maketitle % Insert the title, author and date

\section{Introduction}
\textbf{Part A}
\\
In part A, we write three simple assembly programs to mimic three functions in example.c. 
Based on ensuring correctness,we especially focus on the functional equivalence with the example C functions. 
By selecting and placing labels in the assembly code appropriately, the code is also very readable.\\
\textbf{Part B}
\\
In part B, we modify the HCL file of the SEQ to add a new instruction --- iaddl. 
The following is the roadmap to finish this part:
\begin{itemize}
        \item Clarify the computation process of iadd and write it down at the beginning in seq-full.hcl.
        \item Add any dependence relations of iaddl to all boosigs.
        \item Design the datapath for iaddl (generate control signals for src and dst)
\end{itemize}
\textbf{Part C}
\\
We achieve full scores in the benchmark testing in just 2 hours, 
but we spent 2 more days researching all the potential methods to optimize the performance even further. 
The following is our roadmap:
\begin{itemize}
        \item Change the order of the instruction sequence to avoid data hazard and structure hazards, which leaves $CPI \approx 12$.
        \item Beyond the changes on instructions order, we use loop unrolling to reduce the number of conditional check and registers updating.
        \item Use a binary search tree to find the precise remaining number of loops after several rounds of unrolling to achieve complete unrolling, which leaves $CPI \approx 8.95$
        \item 
\end{itemize}
\textbf{Contribution}
\\
\textbf{Ziqi Zhao} : Part A (coding) \& Part B (coding) \& Part C (coding \& designing) \\
\textbf{Yimin Zhao} : Part A (reviewing) \& Part B (reviewing) \& Part C (designing) \& project report

\section{Experiments}
\subsection{Part A}
\subsubsection{Analysis}
In this part, we are asked to 
[In this part, you should give an overall analysis for the task, 
like difficult point, core technique and so on.]

\subsubsection{Code}

[In this part, you should place your code and make it readable in Microsoft Word, please. Writing necessary comments for codes is a good habit.]

\subsubsection{Evaluation}

[In this part, you should place the figures of experiments for your codes, prove the correctness and validate the performance with your own words for each figure’s explanation.]

\subsection{Part B}

\subsubsection{Analysis}

[In this part, you should give an overall analysis for the task, like difficult point, core technique and so on.]

\subsubsection{Code}

[In this part, you should place your code and make it readable in Latex, please. Writing necessary comments for codes is a good habit.]

\subsubsection{Evaluation}

[In this part, you should place the figures of experiments for your codes, prove the correctness and validate the performance with your own words for each figure’s explanation.]

\subsection{Part C}

\subsubsection{Analysis}

[In this part, you should give an overall analysis for the task, like difficult point, core technique and so on.]

\subsubsection{Code}

[In this part, you should place your code and make it readable in Microsoft Word, please. Writing necessary comments for codes is a good habit.]

\subsubsection{Evaluation}

[In this part, you should place the figures of experiments for your codes, prove the correctness and validate the performance with your own words for each figure’s explanation.]

\section{Conclusion}

\subsection{Problems}

[In this part you can list the obstacles you met during the project, and better add how you overcome them if you have made it.]

\subsection{Achievements}

[In this part you can list the strength of your project solution, like the performance improvement, coding readability, partner cooperation and so on. You can also write what you have learned if you like.]



%----------------------------------------------------------------------------------------


\end{document}